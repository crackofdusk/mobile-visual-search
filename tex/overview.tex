\section{Overview}

The ubiquity of mobile devices equiped with video cameras and Internet access has enabled a new variety of services in which images and video can be used for user input instead of text. One application of this is the possibility of visual search, that is, to identify an object by using an image of it.

Visual search in the mobile context is a process which usually involves two sides: a client and a server. The server's role is to hold a set of images of the things that are avaible for search and to reply to queries from the client, i.e.\ to match an image from the set to the image that the client has. The client doesn't typically send an image directly for comparison. Rather, it does some processing on its side,before sending an alternative representation of the image. This representation is constituted by points or areas of interest in the image which have been encoded in some way. The server holds such representations for every image, having done all the work while populating the database.

If the general description above is vague, it is because many methods have been developped to make visual search possible. This report is a study of the work of Girod et al.\ \cite{girod2011mobile} from which one method has been selected at every step of the process amongst the multitude of possibilities discussed in the paper.

The report is structured as follows: section~\ref{sec:feature_detection} explains how points of interest are found using the so called Fast Hessian detector. Section~\ref{sec:feature_descriptor} presents the encoding and compression of these points of interest by means of the Compressed Histogram of Gradients descriptor. So far, this is work done on the client side. Sections \ref{sec:feature_matching} and \ref{sec:geometric_verification} discuss the work done on the server. Section~\ref{sec:feature_matching} deals with feature matching, i.e.\ how the data base is filtered to produce candidate matches to the descriptor sent by the client. Finally, section~\ref{sec:geometric_verification} explains how the candidate matches are further checked for consistency in a process called geometric verification.
