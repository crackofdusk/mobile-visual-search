\section{Geometric verification}
\label{sec:geometric_verification}

After finding possible matchtes for the input image in the database in the feature matching step of the retrieval, it is important to confirm that the locations of the features are geometrically consistent between the input and the result. This is due to the fact that, while depicting the same scene or object, the two images might differ in viewpoint.

\subsection{Location coding}

Location coding is the concept of representating the positions of the extracted features in the input image. It is done after the feature descriptors have been computed, but it is presented here, because the location information is used in the geometric verification step in the image retrieval process.

An efficient method of location coding is to build then compress a histogram of the locations, as presented by Chen et al.\ in \cite{chen2010inverted}. The authors consider a histogram composed of two pieces which are coded separately: a histogram map and a histogram count.

The histogram map is a subdivision of the image into blocks with binary values. A value of 1 indicates there are features within the boundaries of the block, a value of 0 denotes the lack of features in that area.

The histogram count is the count of features in non emply blocks.

\subsection{Ratio test}

It can happen that a feature detected in the query image is not in the database, because it was not detected during the training. Lowe \cite{Lowe04distinctiveimage} provides a method for eliminating those features, called the ratio test.

From the candidate features, let the nearest neighbour to the query feature be the one which descriptor vector is at minimum distance from the query feature's descriptor vector, noted here $d_1$. Let the second-closest neighbour be the closest neighbour (again using Eucledian distance) that comes from a different object, its distance from the the query feature noted $d_2$. Let $r$ be the ratio $r = \frac{d1}{d2}$.

From empirical testing \cite{Lowe04distinctiveimage}, Lowe has concluded that when the nearest neighbour is a correct match $r$ is distinctly lower than when it is not a correct match. Thus, one can set a threshold ratio and reject matches for which the ratio of distances is larger than the threshold.


\subsection{Random Sample Consensus}

Random Sample Consensus or RANSAC \cite{fischler1981random} is a way of estimating the parameteres of a model by random sampling. It works on the premise that the data set it works with contains points whose distribution can be explained by a set of parameters (these points are called \emph{inliers}) and there are also points which do not fit any model (\emph{outliers}). The algorithm works roughly in the following way.

For a model that needs a minimum of $n$ points to construct its free parameters, select $n$ points at random from the data set. Determine the model parameters using the selected points and compute the set of inliers of that model, i.e. the points that are within a certain error tolerance according to the created model. The model is good enough if the set of inliers is sufficiently large. If it is not, create a new model with a new random subset of the data points and try again. After a predetermined maximum number of tries, select the hypothesis with the largest set of inliers and compute the model parameters using only the inliers.

RANSAC can be used to rank the candidate images by order of consistency with the query image. The ranking critierion is the number of inliers in the estimation of a 2D affine transformation between the query image and the candidate image \cite{ransac_gcc}:
\begin{equation}
\begin{bmatrix}
u_i \\ v_i
\end{bmatrix}
=
\begin{bmatrix}
m_1 & m_2 \\
m_3 & m_4
\end{bmatrix}
\begin{bmatrix}
x_i \\ y_i
\end{bmatrix}
+
\begin{bmatrix}
t_1 \\ t_2
\end{bmatrix},
\end{equation}
where $(x_i,y_i)$ is the location of a query feature and $(u_i,v_i)$ is the location of a candidate feature. The free parameters $m_1,m_2,m_3,m_4$ account for rotation and scaling and $t_1,t_2$ for translation. RANSAC can estimate a solution to the system with 6 unkowns:
\begin{equation}
\begin{bmatrix}
 & & \cdots \\
x_i & y_i & 0 & 0 & 1 & 0 \\
0 & 0 & x_i & y_i & 0 & 1 \\
 & & \cdots \\
\end{bmatrix}
\begin{bmatrix}
m_1 \\ m_2 \\ m_3 \\ m_4 \\ t_1 \\ t_2
\end{bmatrix}
=
\begin{bmatrix}
\vdots \\
u_i \\
v_i \\
\vdots
\end{bmatrix}
\end{equation}

\subsection{Geometric re-ranking}

Tsai et al.\ propose a geometric re-ranking scheme \cite{tsai2010fast} to be used before the geometric verification step in order to improve the accuracy of retrieval. The idea is to apply geometric similarity scores to the candidate images and select only the best ranked images for geometric verification. This is worthwile, because the re-ranking step is significantly faster than the geometric verification step. The authors of \cite{tsai2010fast} evalute geometric similarity scoring based on location, orientation and scale. They retain the location-based ranking for performance reasons.

First, they create a list $M$ of pairs indexes of features from $F_q = \{ l_{q,i}, d_{q,i}\}$ the set of query features characterized by their location $l_q$ and descriptor $d_q$ and $F_d = \{ l_{d,i}, d_{d,i}\}$ the set of database features. A pair of indexes $(m,n)$ is in $M$ if and only if $d_{q,m}$ is the only query descriptor that has visited a node in the vocabulary tree and $d_{d,n}$ is the only candidate descriptor which has visited the same node.

Then they use the location information of the features in the list to create a set $S$ of distance ratios:
\begin{equation}
    S = \left\{ \log \left( \frac{dist(l_{q,i},l_{d,m})}{dist(l_{d,j},l_{d,n})} \right) \big| \; (i,j),(m,n) \in M \right\},
\end{equation}
where $dist(\cdot,\cdot)$ is the Euclidean distance.

The score is given by:
\begin{equation}
    Score = \underset{\alpha}{\max} \sum_{z \in S} I\left( \frac{\alpha}{c} \leq z < \frac{\alpha + 1}{c} \right),
\end{equation}
where, $I(\cdot)$ is the indicator function, $c$ is a tolerance factor and $\frac{\alpha}{c}$ is the scale ratio difference \cite{tsai2010fast}.
